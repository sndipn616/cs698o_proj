%%
%% This is file `tikzposter-example.tex',
%% generated with the docstrip utility.
%%
%% The original source files were:
%%
%% tikzposter.dtx  (with options: `tikzposter-example.tex')
%% 
%% This is a generated file.
%% 
%% Copyright (C) 2014 by Pascal Richter, Elena Botoeva, Richard Barnard, and Dirk Surmann
%% 
%% This file may be distributed and/or modified under the
%% conditions of the LaTeX Project Public License, either
%% version 2.0 of this license or (at your option) any later
%% version. The latest version of this license is in:
%% 
%% http://www.latex-project.org/lppl.txt
%% 
%% and version 2.0 or later is part of all distributions of
%% LaTeX version 2013/12/01 or later.
%% 








 \documentclass[25pt, a0paper, portrait, margin=2mm, innermargin=15mm,
     blockverticalspace=15mm, colspace=15mm, subcolspace=5mm]{tikzposter} %Default values for poster format options.

 \tikzposterlatexaffectionproofon %shows small comment on how the poster was made at bottom of poster

 % Commands
 \newcommand{\bs}{\textbackslash}   % backslash
 \newcommand{\cmd}[1]{{\bf \color{red}#1}}   % highlights command

 % Title, Author, Institute
 \title{Knowledge Distillation and its Variations}
 \author{Sandipan Mandal \& Teekam Chand Mandan}
 \institute{}

 % -- PREDEFINED THEMES ---------------------- %
 % Choose LAYOUT:  Default, Basic, Rays, Simple, Envelope, Wave, Board, Autumn, Desert,
%  \usetheme{Autumn}
\usetheme{Desert}
\usecolorstyle[colorPalette=BlackBlackBlack]{Britain}

\usepackage{multirow}
\usepackage[backend=bibtex]{biblatex}
\addbibresource{cs698o_project.bib}

 \begin{document}

     \maketitle

     \begin{columns}%blocks will be placed into columns
         \column{.50}
         \block[]{Introduction}{In this project, we have implemented Knowledge Distillation(KD) from \cite{hinton2015distilling} and tried some variations of it. We evaluated the performance of KD on MNIST, notMNIST and SVHN datasets. Finally we have also implemented the paper Fitnets : Hints for Thin Deep Nets\cite{romero2014fitnets} and evaluated it on MNIST dataset.
            }
    
     \block[]{Knowledge Distillation}{In KD, we train a large teacher network \textbf{T}. We train our smaller student network \textbf{S} using a composite loss function consisting of two terms, one optimizing to improve label prediction accuracy, other to match pre-softmax logits with teacher \textbf{T}. Loss function is given by : $$L(W_S) = H(y_{true}, P_S) + \lambda H(P_T^{\tau}, P_S^{\tau})$$ where $H$ is the cross-entropy loss function, $y_{true}$ is the true label, $P_S$ is the softmax output of \textbf{S}, $P_T^{\tau}, P_S^{\tau}$ are respectively softmax output of \textbf{T} and \textbf{S} after dividing the corresponding logits by $\tau$.  }
     
  \block[]{Basic Adversarial Knowledge Distillation}{Here, we train a large teacher network \textbf{T} and smaller student network \textbf{S}. Using \textbf{T} and \textbf{S}, we train a discriinator network \textbf{D}, which can distinguish if a logit is constructed from \textbf{T} or \textbf{S}. We train our final student network \textbf{S} using a composite loss function consisting of two terms, one optimizing to improve label prediction accuracy, other to match pre-softmax logits with teacher \textbf{T}. The only difference here is that the logit matching is done adversarially using \textbf{D}. The loss function is given by : $$L(W_S) = \alpha H(y_{true}, P_S) - \beta H(network_{true} , P_D)$$ where $P_D$ is the softmax output of the disriminator network which is fed the student logit as input, $network_{true}$ gives whether the given logit comes from \textbf{T} or \textbf{S}. During training, the logit always comes from \textbf{S}. The other symbols are the same as earlier.}
  
  \block[]{Knowledge Distillation using GANs}{Here we generalize the approach we have taken earlier. It is based on GANs. Though it uses GANs,
the loss function we used is different from the one used by\cite{xu2017learning}. In this approach instead of pre-training the discriminator, we train discriminator, \textbf{D} and student network (generator), \textbf{G} on the go much like GANs. We assume the logits generated from \textbf{T} as \textbf{real} sample and the one generated from \textbf{S} as \textbf{fake} sample. The loss function for \textbf{D} is same as vanilla GAN, but for \textbf{G} which also happens to be student network has an additional term in it. $$L(W_D) = -\gamma[log(P_{real}(x)) + log(1-P_{fake}(x))]$$ $$L(W_G) = \beta H(y_{true}, P_S) - \alpha log(P_{fake}(x)) $$ where $P_{real}(x) = \sigma(D(T(x)))$ and $P_{fake}(x) = \sigma(D(G(x)))$, $\sigma()$ being sigmoid function.}
\block{Knowledge Distillation using Wasserstein GANs}{This aproach is exactly the same as the one above, except that here we use Wasserstein GANs (WGANs) instead of vanilla GANs. As a result the loss function is slightly different. The loss functions are:
$$L(W_D) = -\gamma[f_{real}(x) - f_{fake}(x)]$$ $$L(W_G) = \beta H(y_{true}, P_S) - \alpha f_{fake}(x) $$ where $f_{real}(x) = D(T(x))$ and $f_{fake}(x) = D(G(x))$}

\block[]{Knowledge Distillation (Schematic Diagram)}
{
    \begin{tikzfigure}
        \includegraphics[width=0.3\textwidth, height=0.2\textwidth]{images/KD.png}
    \end{tikzfigure}
}
%     \note[targetoffsetx=24cm, targetoffsety=-9cm,rotate=1,angle=270,radius=8cm,width=.75\textwidth,innersep=.4cm]{
%         You can place notes that are ``attached'' to the previous block using the command
%         \begin{quote} \texttt{\bs note[{\it options}]\{{\it contents}\}}\end{quote}
%         The note is placed by default slightly to the right of a ``target'' in the center of the previous block.  The note style may also allow for a connection between the note and the ``target''.  \\
%         The target may be shifted from the default by setting the options  \texttt{targetoffsetx, targetoffsety}, rotated by an angle with \texttt{rotate}, and its width with \texttt{width}.  The placement of the note in relation with the target is given in polar coordinates with \texttt{ radius, angle}. Please observe that notes are always drawn {\bf over} the other objects. They do not affect the placement of blocks.
%      }

\column{.5}
\block[]{Fitnets : Hints for Thin Deep Nets}{
The approach of this paper\cite{romero2014fitnets} relies on the Knowledge Distillation\cite{hinton2015distilling} approach. This method is used to train a student network which is deeper but also thinner than the teacher network. First a wide and deep teacher network \textbf{T} is trained. Then we decide upon student network \textbf{S} which is thinner but deeper. The training of \textbf{S} proceeds in two phases. The first phase is bringing the intermediate representation of \textbf{T} and \textbf{S} together. The loss function for this step is:
$$L(W_{guided}, W_r) = \frac{1}{2}\|u_T(x,W_{hint}) - r(u_S(x, W_{guided}), W_r) \|^2 $$
where $W_{guided}$ and $W_{hint}$ are correspondingly the set of parameters of \textbf{S} and \textbf{T} upto the middle layer and $W_r$ is the regression weight applied to project the intermediate representation of \textbf{S} into the same dimension as the intermediate representation of \textbf{T}.\\
The second phases of training \textbf{S} is applying Knowledge Distillation on the whole network with the parameters till middle layer initialized with $W_{guided}$.
}

\block[]{Fitnet Training (Schematic Diagram)}
{
    \begin{tikzfigure}
        \includegraphics[width=0.4\textwidth, height=0.32\textwidth]{images/Fitnet.png}
    \end{tikzfigure}
}

\block[]{Experiments (Architecture - I)}{
Exact architecture of \textbf{T} and \textbf{S} are different for different datasets but in each case both of them are fully connected networks with \textbf{S} having typically $~50\%$ as much parameters compared to \textbf{T}.
\begin{tikzfigure}
%   \centering
%   \caption{Experimental Results for architecture - 1}
%   \label{tab:table1}
  \begin{tabular}{| c | c | c | c | c | c |}
    \hline
    \multirow{2}{*}{Dataset} & \multicolumn{4}{c|}{Number of Mis- Classifications} & \multirow{2}{*}{No. of test examples} \\
    \cline{2-5}
    & T & S (initial)  &  S (KD) & S (adver. KD) & \\
    \cline{1-6}
    MNIST & 476 & 597 & 583 & \textbf{375} & 10000\\
    \cline{1-6}
    notMNIST & 1200 & 1344  & 1321 & \textbf{1089} & 10000 \\
    \cline{1-6}
    SVHN & 8339 & 9101 & 9014 & \textbf{8659} & 26032\\
    \cline{1-6}
  \end{tabular}
\end{tikzfigure}
}
\block[]{Experiments (Architecture-II)}{
$\#$ parameters in \textbf{T}: $3250000$, with $2$ conv layers followed by $2$ fc layers.\\
$\#$ parameters in \textbf{S}: $106400$ (Nearly $3.27\%$ of \textbf{T}), with $3$ conv layers followed by $3$ fc layers.
\begin{tikzfigure}
% \begin{table}[h!]
%   \centering
%   \caption{Experimental Results for architecture - 2}
%   \label{tab:table2}
  \begin{tabular}{| c | c | c | c | c | c | c | c |}
    \hline
    \multirow{2}{*}{Dataset} & \multicolumn{7}{c|}{$\#$ Mis- Classifications}  \\
    \cline{2-8}
    & T & S  & KD &  adv. KD & GAN KD & WGAN KD & Guided KD \\
    \cline{1-8}
    MNIST & 611 & 1433 & 809 & 856 & 753 & \textbf{565} & 737 \\
    \cline{1-8}
    notMNIST & 1393 & 2264 & 1760 & 1615 & 1702 &  \textbf{1397}& - \\
    \cline{1-8}
    SVHN & 5199 & 19419 & 16652 & 12964 & 12797 & \textbf{10816} & - \\
    \cline{1-8}
  \end{tabular}
% \end{table}
\end{tikzfigure}
}
        %  \begin{subcolumns}
        %      \subcolumn{.45}
        %      \block{Subcolumns}{If you want to have an additional subdivision of columns inside a column, you may use the\\ \texttt{\bs subcolumns} environment inside of a column environment.  The functionality is similar to that of columns, but now the widths are relative to the width of the current column.}

        %      \subcolumn{.5}
        %      \block{}{An example use of subcolumns is.
        %          \begin{quote}
        %              \texttt{\bs begin\{subcolumns\}\\
        %              \bs subcolumn\{.6\}\\
        %              \bs block\{\dots\}\{\dots\}\\
        %              \bs subcolumn\{.4\}\\
        %              \bs block\{\dots\}\{\dots\}\\
        %              \bs block\{\dots\}\{\dots\}\\
        %              \bs end\{subcolumns\}
        %              }
        %      \end{quote}
        %  }
        %  \end{subcolumns}

         \block[]{References}{\printbibliography[heading=none]}

     \end{columns}

%     \block[titleoffsety=-1cm,bodyoffsety=-1cm]{Sample document}{\vspace{2em}
%         This poster was created by the following commands (omitting the contents of the blocks and notes) to give a sense of how different objects are created and options used.
%         \begin{quote}
%             \texttt{\bs documentclass[25pt, a0paper, portrait, margin=0mm, innermargin=15mm,
%         blockverticalspace=15mm, colspace=15mm, subcolspace=8mm]\{tikzposter\}\\
%             \bs title\{Using tikzposter\} \bs author\{Pascal Richter, Elena Botoeva, Richard Barnard, \& Dirk Surmann\} \bs institute\{\}\\
%              \bs usetheme\{Autumn\}\bs usecolorstyle[colorPalette=BrownBlueOrange]\{Germany\}\\
%             \bs begin\{document\}\bs maketitle\\
%             \bs begin\{columns\} \bs column\{0.55\}\\
%             \bs block\{Creating the document\}\{The document\dots\} \bs note[targetoffsetx=-.05\bs textwidth,targetoffsety=9.5cm,innersep=.4cm,angle=-45,connection]\{\dots\}\\
%             \bs block\{The title matter\}\{The title\dots\}\\
%             \bs block\{Blocks\}\{Blocks are\dots\} \bs note[targetoffsetx=24cm, targetoffsety=-9cm,rotate=1,angle=270,radius=8cm,width=.75\bs textwidth,innersep=.4cm]\{You can\dots\}\\
%             \bs column\{0.45\} \bs block\{Columns\}\{By default,\dots\}\\
%             \bs begin\{subcolumns\} \bs subcolumn\{.45\}
%             \bs block\{Subcolumns\}\{If you\dots\}
%             \bs subcolumn\{.5\} \bs block\{\}\{An example\dots\}
%             \bs end\{subcolumns\}\\
%             \bs block[titlewidthscale=.8,bodywidthscale=.9,titleoffsety=9.5mm,bodyoffsety=9mm]\{Changing the Poster's Appearance\}\{If the default\dots\}
%             \bs end\{columns\}\\
%             \bs block[titleoffsety=-1cm,bodyoffsety=-1cm]\{Sample document\}\{This poster\dots\}\\
%             \bs end\{document\}
%             }
%         \end{quote}
%     }

 \end{document}




\endinput
%%
%% End of file `tikzposter-example.tex'.


